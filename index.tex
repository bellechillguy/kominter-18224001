% Options for packages loaded elsewhere
% Options for packages loaded elsewhere
\PassOptionsToPackage{unicode}{hyperref}
\PassOptionsToPackage{hyphens}{url}
\PassOptionsToPackage{dvipsnames,svgnames,x11names}{xcolor}
%
\documentclass[
  letterpaper,
  DIV=11,
  numbers=noendperiod]{scrreprt}
\usepackage{xcolor}
\usepackage{amsmath,amssymb}
\setcounter{secnumdepth}{5}
\usepackage{iftex}
\ifPDFTeX
  \usepackage[T1]{fontenc}
  \usepackage[utf8]{inputenc}
  \usepackage{textcomp} % provide euro and other symbols
\else % if luatex or xetex
  \usepackage{unicode-math} % this also loads fontspec
  \defaultfontfeatures{Scale=MatchLowercase}
  \defaultfontfeatures[\rmfamily]{Ligatures=TeX,Scale=1}
\fi
\usepackage{lmodern}
\ifPDFTeX\else
  % xetex/luatex font selection
\fi
% Use upquote if available, for straight quotes in verbatim environments
\IfFileExists{upquote.sty}{\usepackage{upquote}}{}
\IfFileExists{microtype.sty}{% use microtype if available
  \usepackage[]{microtype}
  \UseMicrotypeSet[protrusion]{basicmath} % disable protrusion for tt fonts
}{}
\makeatletter
\@ifundefined{KOMAClassName}{% if non-KOMA class
  \IfFileExists{parskip.sty}{%
    \usepackage{parskip}
  }{% else
    \setlength{\parindent}{0pt}
    \setlength{\parskip}{6pt plus 2pt minus 1pt}}
}{% if KOMA class
  \KOMAoptions{parskip=half}}
\makeatother
% Make \paragraph and \subparagraph free-standing
\makeatletter
\ifx\paragraph\undefined\else
  \let\oldparagraph\paragraph
  \renewcommand{\paragraph}{
    \@ifstar
      \xxxParagraphStar
      \xxxParagraphNoStar
  }
  \newcommand{\xxxParagraphStar}[1]{\oldparagraph*{#1}\mbox{}}
  \newcommand{\xxxParagraphNoStar}[1]{\oldparagraph{#1}\mbox{}}
\fi
\ifx\subparagraph\undefined\else
  \let\oldsubparagraph\subparagraph
  \renewcommand{\subparagraph}{
    \@ifstar
      \xxxSubParagraphStar
      \xxxSubParagraphNoStar
  }
  \newcommand{\xxxSubParagraphStar}[1]{\oldsubparagraph*{#1}\mbox{}}
  \newcommand{\xxxSubParagraphNoStar}[1]{\oldsubparagraph{#1}\mbox{}}
\fi
\makeatother


\usepackage{longtable,booktabs,array}
\usepackage{calc} % for calculating minipage widths
% Correct order of tables after \paragraph or \subparagraph
\usepackage{etoolbox}
\makeatletter
\patchcmd\longtable{\par}{\if@noskipsec\mbox{}\fi\par}{}{}
\makeatother
% Allow footnotes in longtable head/foot
\IfFileExists{footnotehyper.sty}{\usepackage{footnotehyper}}{\usepackage{footnote}}
\makesavenoteenv{longtable}
\usepackage{graphicx}
\makeatletter
\newsavebox\pandoc@box
\newcommand*\pandocbounded[1]{% scales image to fit in text height/width
  \sbox\pandoc@box{#1}%
  \Gscale@div\@tempa{\textheight}{\dimexpr\ht\pandoc@box+\dp\pandoc@box\relax}%
  \Gscale@div\@tempb{\linewidth}{\wd\pandoc@box}%
  \ifdim\@tempb\p@<\@tempa\p@\let\@tempa\@tempb\fi% select the smaller of both
  \ifdim\@tempa\p@<\p@\scalebox{\@tempa}{\usebox\pandoc@box}%
  \else\usebox{\pandoc@box}%
  \fi%
}
% Set default figure placement to htbp
\def\fps@figure{htbp}
\makeatother





\setlength{\emergencystretch}{3em} % prevent overfull lines

\providecommand{\tightlist}{%
  \setlength{\itemsep}{0pt}\setlength{\parskip}{0pt}}



 


\KOMAoption{captions}{tableheading}
\makeatletter
\@ifpackageloaded{bookmark}{}{\usepackage{bookmark}}
\makeatother
\makeatletter
\@ifpackageloaded{caption}{}{\usepackage{caption}}
\AtBeginDocument{%
\ifdefined\contentsname
  \renewcommand*\contentsname{Table of contents}
\else
  \newcommand\contentsname{Table of contents}
\fi
\ifdefined\listfigurename
  \renewcommand*\listfigurename{List of Figures}
\else
  \newcommand\listfigurename{List of Figures}
\fi
\ifdefined\listtablename
  \renewcommand*\listtablename{List of Tables}
\else
  \newcommand\listtablename{List of Tables}
\fi
\ifdefined\figurename
  \renewcommand*\figurename{Figure}
\else
  \newcommand\figurename{Figure}
\fi
\ifdefined\tablename
  \renewcommand*\tablename{Table}
\else
  \newcommand\tablename{Table}
\fi
}
\@ifpackageloaded{float}{}{\usepackage{float}}
\floatstyle{ruled}
\@ifundefined{c@chapter}{\newfloat{codelisting}{h}{lop}}{\newfloat{codelisting}{h}{lop}[chapter]}
\floatname{codelisting}{Listing}
\newcommand*\listoflistings{\listof{codelisting}{List of Listings}}
\makeatother
\makeatletter
\makeatother
\makeatletter
\@ifpackageloaded{caption}{}{\usepackage{caption}}
\@ifpackageloaded{subcaption}{}{\usepackage{subcaption}}
\makeatother
\usepackage{bookmark}
\IfFileExists{xurl.sty}{\usepackage{xurl}}{} % add URL line breaks if available
\urlstyle{same}
\hypersetup{
  pdftitle={My Personal Review},
  pdfauthor={131902360 Armein Z R Langi},
  colorlinks=true,
  linkcolor={blue},
  filecolor={Maroon},
  citecolor={Blue},
  urlcolor={Blue},
  pdfcreator={LaTeX via pandoc}}


\title{My Personal Review}
\usepackage{etoolbox}
\makeatletter
\providecommand{\subtitle}[1]{% add subtitle to \maketitle
  \apptocmd{\@title}{\par {\large #1 \par}}{}{}
}
\makeatother
\subtitle{Portfolio Asesmen II-2100 KIPP}
\author{131902360 Armein Z R Langi}
\date{2025-09-15}
\begin{document}
\maketitle

\renewcommand*\contentsname{Table of contents}
{
\hypersetup{linkcolor=}
\setcounter{tocdepth}{2}
\tableofcontents
}

\bookmarksetup{startatroot}

\chapter*{Selamat Berjumpa}\label{selamat-berjumpa}
\addcontentsline{toc}{chapter}{Selamat Berjumpa}

\markboth{Selamat Berjumpa}{Selamat Berjumpa}

\begin{figure}[H]

{\centering \includegraphics[width=9.5\linewidth,height=\textheight,keepaspectratio]{images/AZRL.png}

}

\caption{About Me}

\end{figure}%

Armein Z R Langi adalah Guru Besar di Sekolah Teknik Elektro dan
Informatika ITB, dosen ITB sejak Desember 1987, mantan Rektor
Universitas Kristen Maranatha, 1 Maret 2016 s/d 29 Februari 2020, mantan
Kepala Pusat Penelitian Teknologi Informasi dan Komunikasi (PP-TIK) ITB
November 2005 s/d Maret 2010, dan Sekretaris MWA ITB Mei 2010-Jan 2011.

Lahir di Tomohon 1962 dari pasangan Manado dan Sunda. Saat ini tinggal
di Bandung, menikah dengan Ina dan dikaruniai empat anak. Ayah dari
Gladys, Kezia, Andria, dan Marco.

Sharing pikiran singkat ada di blog \url{https://azrl.wordpress.com}.
Facebookk: armein\_langi

Andria, putri saya, lumayan sering membaca blog ini. Dia juga sumber
inspirasi tulisan saya. Dia senang menceritakan jokes pada saya, dan
kalau bagus saya tulis di sini, terutama seri Kocak Ala Andria. Dan dia
suka heran, karena isinya sudah berbeda. Memang saya senang
mengubah-ngubah cerita karena saya tidak suka menjiplak mentah-mentah.
Minggu lalu tidak sengaja dia memuji tulisan blog ini.

``Papa\ldots{} papa\ldots{} ``, tanyanya, ``Tulisan blog papa itu
copy-paste dari tulisan orang ya\ldots?''

Haa? Ya nggak mungkin lah. Kecuali lirik lagu, semua artikel di sini
ditulis sendiri.

``Ah bohong, soalnya tulisannya terlalu banyak\ldots.'' katanya tetap
tidak percaya, ``Nggak mungkinlah papa tulis sendiri\ldots{}''

Hehe, saya tersenyum sambil membelai Andria, karena buat saya itu adalah
ultimate compliment. Thanks sweetheart\ldots{}

Jadi kalau orang tidak percaya bahwa sesuatu itu karya anda, anda yang
buat, jangan marah. Itu adalah pujian yang sejati.

\bookmarksetup{startatroot}

\chapter{UTS-1 All About Me}\label{uts-1-all-about-me}

\bookmarksetup{startatroot}

\chapter{Aku, di Antara Kode dan Kopi
Pagi}\label{aku-di-antara-kode-dan-kopi-pagi}

Hai, aku Nisrina Zakiyah. Orang-orang bilang aku pendiam, tapi pikiranku
jarang berhenti bekerja. Aku suka membongkar sesuatu, mencari tahu
bagaimana semuanya berjalan---entah itu baris kode, mesin kopi, atau
bahkan cara orang berpikir.

Aku mahasiswa Sistem dan Teknologi Informasi di ITB. Bagiku, teknologi
bukan sekadar alat, tapi cara memahami manusia. Setiap sistem yang
kubangun selalu kutanya: ``Apakah ini membantu orang menjadi lebih
baik?''

Pagi hari adalah waktu favoritku. Sambil menunggu air mendidih untuk
kopi, aku biasanya menulis ide kecil di sticky note, kadang tentang
proyek, kadang tentang hal aneh seperti ``bagaimana kalau AI bisa merasa
bosan?''. Sebagian besar ide itu tidak pernah selesai, tapi setiap
catatan kecil membuatku sadar: aku selalu ingin tahu.

Aku sering terlihat serius, tapi sebenarnya aku mudah tertawa. Aku bisa
ngakak sendiri saat kode jalan setelah 3 jam error karena titik koma
hilang. Aku juga belajar, ternyata hidup kadang mirip debugging---kadang
harus sabar cari kesalahan kecil sebelum semua berjalan baik.

Yang membuatku terus maju bukan keinginan untuk jadi yang paling pintar,
tapi untuk tetap berkembang meski lambat. Aku percaya keunikan bukan
dari hasil besar, tapi dari keberanian untuk jujur pada diri sendiri.

Jadi, kalau kamu bertanya siapa aku: aku adalah kombinasi antara rasa
ingin tahu yang tidak pernah selesai, tawa kecil di tengah error, dan
secangkir kopi pagi yang selalu mengawali hari dengan semangat baru.

\bookmarksetup{startatroot}

\chapter{UTS-2 My Songs for You}\label{uts-2-my-songs-for-you}

\bookmarksetup{startatroot}

\chapter{Lagu Kecil untuk Kamu yang Sedang
Berjuang}\label{lagu-kecil-untuk-kamu-yang-sedang-berjuang}

\textbf{Verse 1}\\
Pernahkah kamu merasa dunia bergerak terlalu cepat,\\
sementara langkahmu masih tertinggal di belakang?\\
Aku tahu rasanya, karena aku juga pernah di sana ---\\
di antara rencana yang gagal dan kopi yang sudah dingin.

\textbf{Chorus}\\
Tapi dengar, kamu tidak sendirian.\\
Setiap orang punya ritmenya sendiri.\\
Kadang lagu hidup terdengar fals,\\
tapi itu justru yang membuatnya nyata.

\textbf{Verse 2}\\
Aku menulis lagu ini bukan dengan nada,\\
tapi dengan napas panjang dan tawa yang kaku.\\
Karena aku percaya, perjuangan yang hening pun pantas dirayakan.\\
Kita tidak perlu panggung besar,\\
cukup hati yang mau mendengarkan diri sendiri.

\textbf{Bridge}\\
Kalau hidup terasa berat,\\
ingat: bahkan gitar butuh disetem ulang untuk bisa berbunyi indah lagi.

\textbf{Final Chorus}\\
Jadi teruslah mainkan lagu itu,\\
meski jari terasa pegal,\\
meski suara serak,\\
karena satu hari nanti,\\
melodi kecilmu akan terdengar oleh dunia.

\begin{center}\rule{0.5\linewidth}{0.5pt}\end{center}

\begin{quote}
Lagu ini bukan hanya untuk orang lain.\\
Ini juga untuk aku,\\
yang belajar bahwa keberanian paling sederhana\\
adalah tetap bernyanyi meski sedang takut.
\end{quote}

\begin{center}\rule{0.5\linewidth}{0.5pt}\end{center}

\subsection{Catatan Reflektif}\label{catatan-reflektif}

Lagu ini menggambarkan perjalanan mental saat seseorang menghadapi
tekanan belajar, rasa gagal, dan kelelahan, namun tetap mencoba bangkit
dengan cara sederhana. Gaya penulisannya bebas dan naratif agar pembaca
bisa ``mendengar'' pesannya meski tanpa nada.

\begin{center}\rule{0.5\linewidth}{0.5pt}\end{center}

\bookmarksetup{startatroot}

\chapter{UTS-3 My Stories for You}\label{uts-3-my-stories-for-you}

\bookmarksetup{startatroot}

\chapter{Tentang Gagal, dan Kenapa Aku Tidak
Berhenti}\label{tentang-gagal-dan-kenapa-aku-tidak-berhenti}

Semester pertama di ITB adalah masa paling melelahkan dalam hidupku.\\
Bukan karena tugas yang banyak, tapi karena aku terus membandingkan diri
dengan orang lain.\\
Setiap kali melihat teman bisa menjelaskan konsep sulit dengan mudah,
aku bertanya:\\
``Kenapa aku tidak seperti mereka?''

Suatu malam, aku hampir menyerah.\\
Aku menatap layar yang penuh error merah---bukan hanya di terminal, tapi
juga di pikiranku.\\
Aku merasa semua usahaku sia-sia.\\
Tapi saat itu, seseorang di lab berkata ringan, ``Kalau gak error, gak
belajar.''\\
Entah kenapa, kalimat sederhana itu menamparku dengan lembut.

Aku mulai melihat kegagalan bukan sebagai bukti ketidakmampuan, tapi
bagian dari proses debug diri.\\
Aku belajar bahwa rasa tidak percaya diri bukan hal yang harus dilawan,
tapi dipahami.\\
Karena dari sanalah kita tahu sisi mana yang perlu diperbaiki.

Beberapa bulan kemudian, aku berhasil menyelesaikan proyek yang dulu
kupikir mustahil.\\
Tidak spektakuler, tapi cukup untuk membuatku tersenyum.\\
Aku sadar, mungkin kemenanganku bukan di hasilnya, tapi di keberanianku
untuk tetap mencoba walau takut.

Sekarang, setiap kali aku menemui error baru---entah di kode atau di
hidup---aku selalu ingat malam itu.\\
Aku tidak ingin sempurna, aku hanya ingin terus belajar.

\begin{center}\rule{0.5\linewidth}{0.5pt}\end{center}

\begin{quote}
Kadang, inspirasi tidak datang dari kemenangan besar.\\
Ia muncul dari momen kecil ketika kamu memilih untuk tidak berhenti.
\end{quote}

\begin{center}\rule{0.5\linewidth}{0.5pt}\end{center}

\subsection{Catatan Reflektif}\label{catatan-reflektif-1}

Cerita ini menggambarkan perjalanan pribadi menghadapi rasa gagal dan
insecurity dalam lingkungan akademik yang kompetitif. Narasi membangun
alur dari keputusasaan ke penerimaan diri, menunjukkan pertumbuhan
emosional yang nyata.

\begin{center}\rule{0.5\linewidth}{0.5pt}\end{center}

\bookmarksetup{startatroot}

\chapter{UTS-4 My SHAPE (Spiritual Gifts, Heart, Abilities, Personality,
Experiences)}\label{uts-4-my-shape-spiritual-gifts-heart-abilities-personality-experiences}

\bookmarksetup{startatroot}

\chapter{Bentukku: Antara Logika dan
Empati}\label{bentukku-antara-logika-dan-empati}

Ketika mengisi lembar kerja tentang ``siapa dirimu'', aku tidak
menyangka jawabannya akan serumit ini.\\
Ternyata, aku bukan hanya seseorang yang berpikir logis, tapi juga
seseorang yang mudah merasakan suasana hati orang lain.\\
Aku menyadari bahwa dua hal itu --- logika dan empati --- adalah bentuk
utamaku.

Di satu sisi, aku suka struktur, data, dan sistem yang rapi.\\
Aku bisa betah berjam-jam di depan laptop, mencoba memahami kenapa
sebuah algoritma tidak berjalan sesuai rencana.\\
Tapi di sisi lain, aku selalu berhenti sejenak saat melihat teman
kelelahan atau diam terlalu lama.\\
Aku ingin tahu apakah mereka baik-baik saja.\\
Dari situ aku sadar, mungkin bentukku bukan sekadar ``problem solver'',
tapi ``listener'' juga.

Dulu aku menganggap perasaan membuatku lemah.\\
Aku kira untuk jadi ``teknolog yang hebat'', aku harus rasional
sepenuhnya.\\
Namun, semakin lama aku belajar, semakin aku sadar: logika tanpa empati
hanyalah kode tanpa konteks.\\
Empati membuat teknologi punya arah.

Sekarang, setiap kali aku membuat sesuatu --- baik kode, desain sistem,
atau tulisan --- aku selalu bertanya,\\
``Apakah ini hanya efisien, atau juga manusiawi?''\\
Karena aku ingin hasil kerjaku tidak hanya berjalan dengan baik, tapi
juga \emph{bermakna bagi orang lain.}

\begin{center}\rule{0.5\linewidth}{0.5pt}\end{center}

\begin{quote}
Bentukku tidak harus sempurna.\\
Yang penting, ia terus berkembang --- mengikuti arah yang membuatku
lebih peka dan lebih bijak.
\end{quote}

\begin{center}\rule{0.5\linewidth}{0.5pt}\end{center}

\subsection{Catatan Reflektif}\label{catatan-reflektif-2}

Tulisan ini menggambarkan pemahaman diri yang seimbang antara kekuatan
analitis dan sisi empatik. Narasi memperlihatkan proses introspeksi dan
kesadaran akan pentingnya keseimbangan nilai dalam pengembangan diri.

\begin{center}\rule{0.5\linewidth}{0.5pt}\end{center}

\bookmarksetup{startatroot}

\chapter{UTS-5 My Personal Reviews}\label{uts-5-my-personal-reviews}

\bookmarksetup{startatroot}

\chapter{Menemukan Diri Lewat
Komunikasi}\label{menemukan-diri-lewat-komunikasi}

Dalam perkuliahan Komunikasi Interpersonal dan Profesional, aku belajar
bahwa komunikasi bukan hanya tentang berbicara, tapi tentang membangun
pemahaman bersama.\\
Sebelumnya aku mengira ``komunikasi yang baik'' berarti mampu
menjelaskan ide dengan jelas. Sekarang aku tahu, itu juga berarti mampu
\emph{mendengarkan} dengan tulus.

Melalui tugas-tugas sebelumnya, aku melihat pola pada diriku sendiri:
aku cenderung fokus pada akurasi pesan, tapi kadang lupa memerhatikan
perasaan orang lain.\\
Di sinilah aku mulai belajar tentang keseimbangan antara \emph{logos}
dan \emph{pathos} --- antara rasionalitas dan empati.

Ketika seseorang menyampaikan pendapat berbeda, refleks awalku adalah
menjawab dengan argumen. Tapi aku sadar, kadang orang tidak butuh
disanggah, hanya butuh dimengerti.\\
Dari sini, aku mulai memperbaiki cara berkomunikasi: bukan untuk menang,
tapi untuk terhubung.

Aku juga menyadari bahwa komunikasi interpersonal tidak bisa dipisahkan
dari etika.\\
Etos komunikasi muncul ketika seseorang berbicara dengan integritas.\\
Bukan sekadar menyampaikan pesan, tapi juga menghargai lawan bicara.\\
Karena itu, aku berusaha lebih hati-hati memilih kata, terutama dalam
situasi yang emosional atau sensitif.

Dari seluruh proses ini, aku menyimpulkan bahwa komunikasi yang efektif
selalu berawal dari kesadaran diri.\\
Saat aku memahami perasaanku sendiri, aku bisa lebih mudah memahami
orang lain.\\
Dan ketika aku mendengarkan dengan empati, aku tidak hanya mendengar
suara --- aku memahami makna.

\begin{center}\rule{0.5\linewidth}{0.5pt}\end{center}

\subsection{Rekomendasi Pribadi}\label{rekomendasi-pribadi}

\begin{enumerate}
\def\labelenumi{\arabic{enumi}.}
\tightlist
\item
  \textbf{Untuk diri sendiri:} Latih keberanian berbicara tanpa
  kehilangan kelembutan dalam penyampaian.\\
\item
  \textbf{Untuk lingkungan belajar:} Ciptakan ruang diskusi yang aman,
  di mana setiap orang merasa didengar tanpa takut salah.\\
\item
  \textbf{Untuk komunikasi profesional:} Gunakan empati sebagai dasar
  dalam setiap presentasi atau interaksi kerja --- karena ide hebat
  hanya berarti bila orang lain merasa dihargai.
\end{enumerate}

\begin{center}\rule{0.5\linewidth}{0.5pt}\end{center}

\begin{quote}
Komunikasi terbaik bukan tentang siapa yang paling fasih, tapi siapa
yang paling tulus dalam memahami.
\end{quote}

\begin{center}\rule{0.5\linewidth}{0.5pt}\end{center}

\bookmarksetup{startatroot}

\chapter{UAS-1 My Concepts}\label{uas-1-my-concepts}

Mau hidup epik ? \href{lifestory.pdf}{Write your Life Story}

Apa itu berkonsep?

\url{https://youtu.be/QVfUlVBO80U?si=yM6q_rwV9rcDBbu7}

\bookmarksetup{startatroot}

\chapter{UAS-3 My Opinions}\label{uas-3-my-opinions}

SApa itu beropini? \href{BM\%20Opini.mp4}{Opini Berpengaruh}

Bagiamana menjaadi menarik? \href{./Interesting.mp4}{Menjadi Menarik}

\bookmarksetup{startatroot}

\chapter{UAS-3 My Innovations}\label{uas-3-my-innovations}

\bookmarksetup{startatroot}

\chapter{UAS-4 My Knowledge}\label{uas-4-my-knowledge}

Cara saya mengkomunikasikan sebuah pengatahuan sebagai petunjuk bagi
orang lain 1) saya tulis
\href{Rekomendasi\%20Presentasi\%20Efektif(Contoh\%20Makalah).pdf}{makalah
sebagai bahan utama} 2) lalu saya buat
\href{Contoh\%20Transkrip\%20Presentasi.pdf}{transkrip ucapan lisan} 3)
kemudian saya kembangkan
\href{Rekomendasi\%20Presentasi\%20(Contoh\%20Slides).pdf}{slide
pendukung trnsskrip} 4) lalu saya memproduksivideo audio visual
\url{https://youtu.be/ZbghfMvnPZc} \url{https://youtu.be/ZbghfMvnPZc}

\bookmarksetup{startatroot}

\chapter{UAS-5 My Professional
Reviews}\label{uas-5-my-professional-reviews}

Untuk melAkukan review, seperti pada
\href{../My_Personal_Reviews/Doc.5.Mengevaluasi-Esai-Berdasarkan-Rubrik.pdf}{pendekatan
AI}, kita membutuhkan rubrik

\bookmarksetup{startatroot}

\chapter{Summary}\label{summary}

In summary, this book has no content whatsoever.

\bookmarksetup{startatroot}

\chapter*{References}\label{references}
\addcontentsline{toc}{chapter}{References}

\markboth{References}{References}

\phantomsection\label{refs}




\end{document}
