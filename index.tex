\hypersetup{
  pdftitle={Identitasku},
  pdfauthor={19624039 Nisrina Zakiyah},
  colorlinks=true,
  linkcolor={blue},
  filecolor={Maroon},
  citecolor={Blue},
  urlcolor={Blue},
  pdfcreator={LaTeX via pandoc}}

\title{Identitasku}
\subtitle{Portfolio Asesmen II-2100 Komunikasi Interpersonal dan Profesional}
\author{19624039 Nisrina Zakiyah}
\date{2025-10-20}

\begin{document}
\maketitle

\renewcommand*\contentsname{Daftar Isi}
{
\hypersetup{linkcolor=}
\setcounter{tocdepth}{2}
\tableofcontents
}

\bookmarksetup{startatroot}

\chapter*{Identitasku}\label{identitasku}
\addcontentsline{toc}{chapter}{Identitasku}

\markboth{Identitasku}{Identitasku}

\begin{figure}[H]
{\centering \includegraphics[width=8cm]{images/nisrina.jpg}}
\caption{Nisrina Zakiyah}
\end{figure}

Halo, aku **Nisrina Zakiyah**, mahasiswa **Sistem dan Teknologi Informasi ITB**.  
Aku suka memahami bagaimana teknologi bisa membantu manusia — bukan hanya sekadar bekerja lebih cepat, tapi juga hidup lebih baik.  
Aku tertarik pada **pengembangan Ground Control Station (GCS)** dan **keamanan siber**, dua bidang yang membuatku sadar bahwa logika dan empati bisa berjalan bersama.  

Di luar kampus, aku menikmati waktu dengan membaca, menulis, dan bereksperimen dengan proyek kecil.  
Aku percaya setiap ide, sekecil apa pun, bisa tumbuh jika diberi ruang.  
Inilah alasanku menulis portofolio ini: bukan untuk menunjukkan kesempurnaan, tapi untuk mencatat prosesku memahami diri sendiri.

---

\bookmarksetup{startatroot}

\chapter{UTS-1 Aku, di Antara Kode dan Kopi Pagi}\label{uts-1}

Hai, aku Nisrina Zakiyah. Orang-orang bilang aku pendiam, tapi pikiranku jarang berhenti bekerja.
% (isi lengkap UTS-1 tetap sama seperti sebelumnya)

---

\chapter{UTS-2 Lagu Kecil untuk Kamu yang Sedang Berjuang}\label{uts-2}
% (isi lengkap UTS-2 seperti sebelumnya)

---

\chapter{UTS-3 Tentang Gagal, dan Kenapa Aku Tidak Berhenti}\label{uts-3}
% (isi lengkap UTS-3 seperti sebelumnya)

---

\chapter{UTS-4 Bentukku: Antara Logika dan Empati}\label{uts-4}
% (isi lengkap UTS-4 seperti sebelumnya)

---

\chapter{UTS-5 Menemukan Diri Lewat Komunikasi}\label{uts-5}
% (isi lengkap UTS-5 seperti sebelumnya)

\bookmarksetup{startatroot}

\chapter*{Referensi}\label{referensi}
\addcontentsline{toc}{chapter}{Referensi}

\markboth{Referensi}{Referensi}
\phantomsection\label{refs}

\end{document}
